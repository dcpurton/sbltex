% \iffalse meta-comment
%
% Copyright (c) 2026 David Purton <dcpurton@marshwiggle.net>
%
% This work may be distributed and/or modified under the conditions of
% the LaTeX Project Public License, either version 1.3c of this license
% or (at your option) any later version. The latest version of this
% license is in
%    http://www.latex-project.org/lppl.txt
% and version 1.3c or later is part of all distributions of LaTeX
% version 2005/12/01 or later.
%
%<*driver>
\RequirePackage{pdfmanagement}
\documentclass[a4paper]{ltxdoc}
\usepackage{microtype}
\usepackage{mlmodern}
\usepackage{multicol}
\usepackage{enumitem}
\usepackage[font=small, skip=6pt]{caption}
\usepackage{graphicx}
\usepackage{xcolor}
\usepackage{listings}
\usepackage{bibleref-sbl}

\newcommand{\testbook}[1]{%
  \item[\bibleverse{#1}] {\brsblfullname\bibleverse{#1}}
}

\setlist[description]{nosep, font=\normalfont, labelwidth=2cm, leftmargin=3cm,
  topsep=\medskipamount}

\AddToHook{env/macrocode/before}{%
  \addvspace{\medskipamount}}

\AddToHook{env/macro/before}{%
  \addvspace{\medskipamount}}

\begin{document}
\DocInput{\jobname.dtx}
\end{document}
%</driver>
% \fi
%
% \title{The \pkg{bibleref-sbl} Package}
% \author{David Purton\thanks{Email: \url{dcpurton@marshwiggle.net}}}
% \date{2026/01/12 v1.0}
%
% \maketitle
%
% \begin{abstract}
%   The \pkg{bibleref-sbl} package extends the \pkg{bibleref} package to
%   support Bible book names as specified by the \emph{Society of Biblical
%   Literature Handbook of Style}.\footnote{\emph{The SBL Handbook of Style:
%   For Biblical Studies and Related Disciplines}, 2nd ed.\@ (Atlanta: SBL
%   Press, 2014).} This includes adjustment of abbreviations, addition of
%   extra Deuterocanonical books, an interface for acceptable options and
%   correct index sorting.
% \end{abstract}
%
% \tableofcontents
%
% \section{Introduction}
%
% \subsection{Usage}
%
% For usage, see the \pkg{bibleref} documentation. This package also works
% with \pkg{bibleref-parse}. Package options and commands unique to
% \pkg{bibleref-sbl} are documented below.
%
% Abbreviated book names have been added to ensure that all SBL abbreviations
% can be used with \cs{bibleverse} and \cs{ibibleverse}, though spaces should
% be omitted. See Table \ref{examplereferences}.
%
% \begin{table}
%   \centering
%   \caption{Example Bible references}\label{examplereferences}
%   \begin{tabular}{ll}
%     |\bibleverse{1Sam}(1:1)| & \bibleverse{1Sam}(1:1) \\
%     |\bibleverse{Ps}(8:)| & \bibleverse{Ps}(8:) \\
%     |\bibleverse{Pss}(1-2:)| & \bibleverse{Pss}(1-2:) \\
%     |\bibleverse{SgThree}(:2-4)| & \bibleverse{SgThree}(:2-4) \\
%     |\bibleverse{Ps151}(:3)| & \bibleverse{Ps151}(:3) \\
%     |\bibleverse{1Cor}(10:31-11:1)| & \bibleverse{1Cor}(10:31-11:1) \\
%   \end{tabular}
% \end{table}
%
% \subsection{Bug Reports and Feature Requests}
%
% Bug reports and feature requests can be made at the \pkg{sbltex} GitHub
% repository. See \url{https://github.com/dcpurton/sbltex}.
%
% \section{Package Options}
%
% \begin{macro}{cant}
%   \SetKeys[brsbl]{cant}
%   Use Canticles instead of Song of Songs, e.g.,
%   \begin{description}
%     \testbook{Cant}
%   \end{description}
% \end{macro}
%
% \begin{macro}{eccl}
%   \SetKeys[brsbl]{eccl}
%   Use Ecclesiastes instead of Qoheleth, e.g.,\hfill\emph{(Default)}
%   \begin{description}
%     \testbook{Eccl}
%   \end{description}
% \end{macro}
%
% \begin{macro}{englishorder}
%   Use English order for Old Testament books in index.\hfill\emph{(Default)}
%
%   \medskip
%
%   \noindent This option sets \cs{brsblot} to `Old Testament/Hebrew Bible'.
% \end{macro}
%
% \begin{macro}{hebreworder}
%   Use Hebrew order for Old Testament books in index.
%
%   \medskip
%
%   \noindent This option sets \cs{brsblot} to `Hebrew Bible/Old Testament'.
% \end{macro}
%
% \begin{macro}{makeindexsorting}
%   Set up Bible book order sorting for |makeindex|.\hfill\emph{(Default)}
%
%   \medskip
%
%   \noindent This option sets up sorting for |makeindex| so Bible books sort
%   in Bible book order rather than alphabetical order. You might set this to
%   |false| if you are using |xindy| for sorting.
%
%   This options must be set at load time.
% \end{macro}
%
% \begin{macro}{qoh}
%   \SetKeys[brsbl]{qoh}
%   Use Qoheleth instead of Ecclesiastes, e.g.,
%   \begin{description}
%     \testbook{Qoh}
%   \end{description}
% \end{macro}
%
% \begin{macro}{song}
%   \SetKeys[brsbl]{song}
%   Use Song of Songs instead of Canticles, e.g.,\hfill\emph{(Default)}
%   \begin{description}
%     \testbook{Song}
%   \end{description}
% \end{macro}
%
% \section{Commands}
%
% \begin{macro}{\brsblabbrvname}
%   Activate SBL abbreviations for book names.
% \end{macro}
%
% \begin{macro}{\brsblfullname}
%   Activate SBL full names for book names.
% \end{macro}
%
% \begin{macro}{\brsbldefaults}
%   Set SBL default format and delimiters.
% \end{macro}
%
% \begin{macro}{\brsblot, \brsbldb, \brsblnt}
%   Name of categories for Old Testament, Deuterocanonical books and New
%   Testament books used in index. They default to `Old Testament/Hebrew
%   Bible', `Deuterocanonical Books' and `New Testament' respectively. If the
%   option |hebreworder| is set then \cs{brsblot} defaults to `Hebrew
%   Bible/Old Testament'.
% \end{macro}
%
% \begin{macro}{\Bibleverse, \iBibleverse}
%   At the beginning of a sentence the full book name should be used rather
%   than an abbreviation. \cs{Bibleverse} and \cs{iBibleverse} provide
%   variants of \cs{bibleverse} and \cs{ibibleverse} to handle this.
% \end{macro}
%
% \begin{macro}{\pBibleverse, \piBibleverse}
%   If \pkg{bibleref-parse} is loaded define \cs{pBibleverse} and
%   \cs{piBibleverse} which use \cs{Bibleverse} and \cs{iBibleverse}
%   respectively.
% \end{macro}
%
% \section{Supported Book Names}
%
% \subsection{\brsblot}
%
% \begin{multicols}{2}
%   \begin{description}
%     \testbook{Gen}
%     \testbook{Exod}
%     \testbook{Lev}
%     \testbook{Num}
%     \testbook{Deut}
%     \testbook{Josh}
%     \testbook{Judg}
%     \testbook{Ruth}
%     \testbook{1Sam}
%     \testbook{1Kgdms}
%     \testbook{2Sam}
%     \testbook{2Kgdms}
%     \testbook{1Kgs}
%     \testbook{3Kgdms}
%     \testbook{2Kgs}
%     \testbook{4Kgdms}
%     \testbook{1Chr}
%     \testbook{2Chr}
%     \testbook{Ezra}
%     \testbook{Neh}
%     \testbook{Esth}
%     \testbook{Job}
%     \testbook{Ps}
%     \testbook{Pss}
%     \testbook{Prov}
%     \testbook{Eccl}
%     \testbook{Song}
%     \testbook{Isa}
%     \testbook{Jer}
%     \testbook{Lam}
%     \testbook{Ezek}
%     \testbook{Dan}
%     \testbook{Hos}
%     \testbook{Joel}
%     \testbook{Amos}
%     \testbook{Obad}
%     \testbook{Jonah}
%     \testbook{Mic}
%     \testbook{Nah}
%     \testbook{Hab}
%     \testbook{Zeph}
%     \testbook{Hag}
%     \testbook{Zech}
%     \testbook{Mal}
%   \end{description}
% \end{multicols}
%
% \subsection{\brsbldb}
%
% \begin{multicols}{2}
%   \begin{description}
%     \testbook{Tob}
%     \testbook{Jdt}
%     \testbook{AddEsth}
%     \testbook{Wis}
%     \testbook{Sir}
%     \testbook{Bar}
%     \testbook{EpJer}
%     \testbook{AddDan}
%     \testbook{PrAzar}
%     \testbook{SgThree}
%     \testbook{Sus}
%     \testbook{Bel}
%     \testbook{1Macc}
%     \testbook{2Macc}
%     \testbook{1Esd}
%     \testbook{PrMan}
%     \testbook{Ps151}
%     \testbook{3Macc}
%     \testbook{2Esd}
%     \testbook{4Macc}
%   \end{description}
%  \end{multicols}
%
% \subsection{\brsblnt}
%
% \begin{multicols}{2}
%   \begin{description}
%     \testbook{Matt}
%     \testbook{Mark}
%     \testbook{Luke}
%     \testbook{John}
%     \testbook{Acts}
%     \testbook{Rom}
%     \testbook{1Cor}
%     \testbook{2Cor}
%     \testbook{Gal}
%     \testbook{Eph}
%     \testbook{Phil}
%     \testbook{Col}
%     \testbook{1Thess}
%     \testbook{2Thess}
%     \testbook{1Tim}
%     \testbook{2Tim}
%     \testbook{Titus}
%     \testbook{Phlm}
%     \testbook{Heb}
%     \testbook{Jas}
%     \testbook{1Pet}
%     \testbook{2Pet}
%     \testbook{1John}
%     \testbook{2John}
%     \testbook{3John}
%     \testbook{Jude}
%     \testbook{Rev}
%   \end{description}
% \end{multicols}
%
% \section{Indexing}
%
% By default books will sort in Bible book order with |makeindex|. This can be
% disabled by passing |makeindexsorting=false| to the \pkg{bibleref-sbl}
% package at load time.
%
% Sorting is set up using the \cs{biblerefmap} command and using a following
% format:
%
% \begin{center}
%   |\biblerefmap{|\meta{book name}|}{|\meta{category order}|@|\meta{category
%   text}|!|\meta{book order}|}|
% \end{center}
% where
% \begin{center}
%   \begin{tabular}{rcc}
%     & \meta{category order} & \meta{category text} \\
%     Old Testament/Hebrew Bible & 10 & \cs{brsblot} \\
%     Deuterocanonical Books & 20 & \cs{brsbldb} \\
%     New Testament & 30 & \cs{brsblnt} \\
%   \end{tabular}
% \end{center}
% So, for Genesis:
% \begin{center}
%   |\biblerefmap{Gen}{10@\brsblot !01}|
% \end{center}
% And Revelation:
% \begin{center}
%   |\biblerefmap{Rev}{30@\brsblnt !27}|
% \end{center}
% This allows other common categories like \emph{Ancient Near Eastern Texts}
% or \emph{Dead Sea Scrolls} to be inserted in between in an Index of Ancient
% Sources using other values for \meta{category order}.
%
% A simple index could be implemented as shown in Figure
% \ref{exampleindexsource}. The output is in Figure \ref{exampleindexoutput}.
% In this example a custom \texttt{ist} file is used to use |\quad\hfill| as
% the delimeter between the reference and the page number rather than the
% default comma. A separate index file is set up for \texttt{makeindex} and
% configured for \pkg{bibleref} by redefining the \cs{biblerefindex} command.
%
% Note that SBL requires page numbers to be compressed and references in
% footnotes to be indicated with n.\ and nn. Automating this requires a
% significant amount of complex code. See the \pkg{sblidx} package for a full
% SBL index implementation.
%
% \lstset{
%   language={[LaTeX]TeX},
%   morekeywords={\biblerefindex, \idxn, \idxnn, \printindex, \ibibleverse},
%   basicstyle=\small\ttfamily,
%   keywordstyle=\color{blue!70!black},
%   frame=single,
%   tabsize=2,
% }
%
% \begin{figure}[b!]
% \iffalse
%<*example>
% \fi
    \begin{lstlisting}
\documentclass{article}
\begin{filecontents}[overwrite]{\jobname.ist}
delim_2 "\\quad\\hfill"
\end{filecontents}

\usepackage{imakeidx}
\usepackage[font=small, itemlayout=relhang, subindent=0pt,
  columns=1, subsubindent=0.25in, hangindent=0.25in,
  initsep=0pt plus 2pt]{idxlayout}
\makeindex[name=\jobname-ancient-sources,
  title=Ancient Souces Index, options=-s \jobname.ist -q]

\usepackage{bibleref-sbl}
\renewcommand{\biblerefindex}{%
  \index[\jobname-ancient-sources]}

\newcommand*{\idxn}[2]{#2 n.~#1}
\newcommand*{\idxnn}[2]{#2 nn.~#1}

\begin{document}
\setcounter{page}{30}\ibibleverse{Gen}(22:20-22)
\clearpage
\setcounter{page}{47}
\ibibleverse[idxnn{15--16}]{Gen}(22:20-22)
\clearpage
\setcounter{page}{57}
\ibibleverse[idxnn{8--9 and 12}]{Exod}(2:3)
\clearpage
\setcounter{page}{63}\ibibleverse{Exod}(2:3)
\clearpage
\setcounter{page}{70}\ibibleverse{Gen}(1:2)
\clearpage
\setcounter{page}{113}\ibibleverse[(]{Gen}(1:7)
\clearpage
\setcounter{page}{115}\ibibleverse[)]{Gen}(1:7)
\clearpage
\setcounter{page}{173}\ibibleverse{Gen}(1:2)
\clearpage
\setcounter{page}{204}\ibibleverse[idxn{3}]{Gen}(1:2)
\clearpage
\printindex[\jobname-ancient-sources]
\end{document}
    \end{lstlisting}
% \iffalse
%</example>
% \fi
%   \caption{Example Ancient Sources Index \TeX\ Source}
%   \label{exampleindexsource}
% \end{figure}
%
% \begin{figure}
%   \centering
%   \fbox{\includegraphics{example-ancient-sources-index.pdf}}
%   \caption{Example Ancient Sources Index Output}
%   \label{exampleindexoutput}
% \end{figure}
%
% \section{Implementation}
%
% \setlength{\parindent}{0em}
%
%    \begin{macrocode}
%<*package>
%    \end{macrocode}
%
%    \begin{macrocode}
\NeedsTeXFormat{LaTeX2e}
\ProvidesPackage{bibleref-sbl}
  [2026/01/12 v1.0 Society of Biblical Literature Bibleref Style (DCP)]
%    \end{macrocode}
%
%    \begin{macrocode}
\RequirePackage{bibleref}
%    \end{macrocode}
%
% Patch \cs{@bible@verse} to handle Psalm 151 which requires \cs{BRbkchsep} to
% be \cs{BRchvsep} in order to print correctly.
%
%    \begin{macrocode}
\patchcmd{\@bible@verse}
  {\BRbkchsep}
  {\ifstrequal{#1}{Ps151}
     {\BRchvsep}
     {\ifstrequal{#1}{PsalmCLI}
        {\BRchvsep}
        {\BRbkchsep}}}
  {}
  {}
%    \end{macrocode}
%
% \begin{macro}{\Bibleverse, \iBibleverse}
%
%   At the beginning of a sentence the full book name should be used rather
%   than an abbreviation. \cs{Bibleverse} and \cs{iBibleverse} provide
%   variants of \cs{bibleverse} and \cs{ibibleverse} to handle this.
%
%    \begin{macrocode}
\DeclareRobustCommand*{\Bibleverse}{%
  \let\@bv@restore\relax
  \begingroup
  \brsblfullname
  \AddToHookNext{cmd/@end@@bibleverse/after}{\endgroup}%
  \@bible@verse
}
%    \end{macrocode}
%
%    \begin{macrocode}
\DeclareRobustCommand*{\iBibleverse}[1][\bvidxpgformat]{%
  \def\@bv@idxpgformat{#1}%
  \@bv@idxtrue
  \def\@bv@idxentry{}%
  \Bibleverse
}
%    \end{macrocode}
% \end{macro}
%
% \begin{macro}{\pBibleverse, \piBibleverse}
%
%   If \pkg{bibleref-parse} is loaded define \cs{pBibleverse} and
%   \cs{piBibleverse} which use \cs{Bibleverse} and \cs{iBibleverse}
%   respectively.
%
%    \begin{macrocode}
\AtEndPreamble{
  \@ifpackageloaded{bibleref-parse}
    {\newcommand{\pBibleverse}[2][]{%
       \brp@parse[#1]{#2}%
       \brp@convert
       \let\brp@do\Bibleverse
       \brp@brlist}
     \newcommand{\piBibleverse}[2][]{%
       \brp@parse[#1]{#2}%
       \brp@convert
       \let\brp@do\iBibleverse
       \brp@brlist}}
    {}
}
%    \end{macrocode}
% \end{macro}
%
% Set up package options.
%
%    \begin{macrocode}
\newif\if@brsbl@cant
\newif\if@brsbl@hebreworder
\newif\if@brsbl@qoh
\newif\if@brsbl@makeindexsorting
\@brsbl@makeindexsortingtrue
\DeclareKeys[brsbl]
  {
    cant.if                = @brsbl@cant ,
    eccl.ifnot             = @brsbl@qoh  ,
    englishorder.ifnot     = @brsbl@hebreworder ,
    hebreworder.if         = @brsbl@hebreworder ,
    makeindexsorting.if    = @brsbl@makeindexsorting ,
    makeindexsorting.usage = load ,
    qoh.if                 = @brsbl@qoh  ,
    song.ifnot             = @brsbl@cant ,
  }
\ProcessKeyOptions[brsbl]
%    \end{macrocode}
%
% Add missing SBL Bible books.
%
%    \begin{macrocode}
\addbiblebook{IKingdoms}{\BRbooknumberstyle{1}Kingdoms}
\addbiblebook{IIKingdoms}{\BRbooknumberstyle{2}Kingdoms}
\addbiblebook{IIIKingdoms}{\BRbooknumberstyle{3}Kingdoms}
\addbiblebook{IVKingdoms}{\BRbooknumberstyle{4}Kingdoms}
\addbiblebook{Psalm}{Psalm}
\addbiblebook{AdditionstoEsther}{Additions to Esther}
\addbiblebook{EpistleofJeremiah}{Epistle of Jeremiah}
\addbiblebook{AdditionstoDaniel}{Additions to Daniel}
\addbiblebook{PrayerofAzariah}{Prayer of Azariah}
\addbiblebook{SongoftheThreeYoungMen}{Song of the Three Young Men}
\addbiblebook{Susanna}{Susanna}
\addbiblebook{BelandtheDragon}{Bel and the Dragon}
\addbiblebook{IEsdras}{\BRbooknumberstyle{1}Esdras}
\addbiblebook{PrayerofManasseh}{Prayer of Manasseh}
\addbiblebook{PsalmCLI}{Psalm 151}
\addbiblebook{IIIMaccabees}{\BRbooknumberstyle{3}Maccabees}
\addbiblebook{IIEsdras}{\BRbooknumberstyle{2}Esdras}
\addbiblebook{IVMaccabees}{\BRbooknumberstyle{4}Maccabees}
%    \end{macrocode}
%
% Add missing SBL abbreviations for use in \meta{book title} arguments.
%
% \begin{macro}{\brsbl@provideabbrv}\marg{abbrev}\marg{book}
%
%   \medskip
%
%   Macro to test if a book abbreviation is defined and if not define it.
%
%    \begin{macrocode}
\newcommand*{\brsbl@provideabbrv}[2]{%
  \@ifundefined{br@#1}{%
    \expandafter\def\csname br@#1\endcsname{\csname br@#2\endcsname}%
  }{}%
}
%    \end{macrocode}
% \end{macro}
%
% \begin{macro}{\brsbl@setabbrv}\marg{abbrev}\marg{book}
%
%   \medskip
%
%   Macro to set an abbreviation. Unlike \cs{brsbl@provideabbrv} this macro
%   overwrites an existing definition.
%
%    \begin{macrocode}
\newcommand*{\brsbl@setabbrv}[2]{%
  \expandafter\def\csname br@#1\endcsname{\csname br@#2\endcsname}%
}
%    \end{macrocode}
% \end{macro}
%
%    \begin{macrocode}
\brsbl@provideabbrv{1Sam}{ISamuel}
\brsbl@provideabbrv{1Kgdms}{IKingdoms}
\brsbl@provideabbrv{2Sam}{IISamuel}
\brsbl@provideabbrv{2Kgdms}{IIKingdoms}
\brsbl@provideabbrv{1Kgs}{IKings}
\brsbl@provideabbrv{3Kgdms}{IIIKingdoms}
\brsbl@provideabbrv{2Kgs}{IIKings}
\brsbl@provideabbrv{4Kgdms}{IVKingdoms}
\brsbl@provideabbrv{1Chr}{IChronicles}
\brsbl@provideabbrv{2Chr}{IIChronicles}
\brsbl@setabbrv{Ps}{Psalm}
\brsbl@provideabbrv{Pss}{Psalms}
\brsbl@provideabbrv{Qoh}{Ecclesiastes}
\brsbl@provideabbrv{Cant}{SongofSongs}
%    \end{macrocode}
%
%    \begin{macrocode}
\brsbl@provideabbrv{Tob}{Tobit}
\brsbl@provideabbrv{AddEsth}{AdditionstoEsther}
\brsbl@provideabbrv{EpJer}{EpistleofJeremiah}
\brsbl@provideabbrv{AddDan}{AdditionstoDaniel}
\brsbl@provideabbrv{PrAzar}{PrayerofAzariah}
\brsbl@provideabbrv{SgThree}{SongoftheThreeYoungMen}
\brsbl@provideabbrv{Sus}{Susanna}
\brsbl@provideabbrv{Bel}{BelandtheDragon}
\brsbl@provideabbrv{1Macc}{IMaccabees}
\brsbl@provideabbrv{2Macc}{IIMaccabees}
\brsbl@provideabbrv{1Esd}{IEsdras}
\brsbl@provideabbrv{PrMan}{PrayerofManasseh}
\brsbl@provideabbrv{Ps151}{PsalmCLI}
\brsbl@provideabbrv{3Macc}{IIIMaccabees}
\brsbl@provideabbrv{2Esd}{IIEsdras}
\brsbl@provideabbrv{4Macc}{IVMaccabees}
%    \end{macrocode}
%
%    \begin{macrocode}
\brsbl@provideabbrv{1Cor}{ICorinthians}
\brsbl@provideabbrv{2Cor}{IICorinthians}
\brsbl@provideabbrv{1Thess}{IThessalonians}
\brsbl@provideabbrv{2Thess}{IIThessalonians}
\brsbl@provideabbrv{1Tim}{ITimothy}
\brsbl@provideabbrv{2Tim}{IITimothy}
\brsbl@provideabbrv{Phlm}{Philemon}
\brsbl@provideabbrv{1Pet}{IPeter}
\brsbl@provideabbrv{2Pet}{IIPeter}
\brsbl@provideabbrv{1John}{IJohn}
\brsbl@provideabbrv{2John}{IIJohn}
\brsbl@provideabbrv{3John}{IIIJohn}
%    \end{macrocode}
%
% \begin{macro}{\brsblfullname}
%
%   Define new \cs{brsblfullname} macro which adjusts \cs{brfullname} book
%   names to bring them into line with SBL requirements.
%
%    \begin{macrocode}
\newcommand*\brsblfullname{%
  \brfullname
  \brsbldefaults
  \setbooktitle{IKingdoms}{\BRbooknumberstyle{1}Kingdoms}%
  \setbooktitle{IIKingdoms}{\BRbooknumberstyle{2}Kingdoms}%
  \setbooktitle{IIIKingdoms}{\BRbooknumberstyle{3}Kingdoms}%
  \setbooktitle{IVKingdoms}{\BRbooknumberstyle{4}Kingdoms}%
  \setbooktitle{Psalm}{Psalm}%
  \setbooktitle{Psalms}{Psalms}%
  \setbooktitle{Ecclesiastes}{%
    \BRbookof\if@brsbl@qoh Qoheleth\else Ecclesiastes\fi}%
  \setbooktitle{SongofSongs}{%
    \BRbookof\if@brsbl@cant Canticles\else Song of Songs\fi}%
  \setbooktitle{AdditionstoEsther}{Additions to Esther}%
  \setbooktitle{Ecclesiasticus}{Sirach}%
  \setbooktitle{EpistleofJeremiah}{Epistle of Jeremiah}%
  \setbooktitle{AdditionstoDaniel}{Additions to Daniel}%
  \setbooktitle{PrayerofAzariah}{Prayer of Azariah}%
  \setbooktitle{SongoftheThreeYoungMen}{Song of the Three Young Men}%
  \setbooktitle{Susanna}{Susanna}%
  \setbooktitle{BelandtheDragon}{Bel and the Dragon}%
  \setbooktitle{IEsdras}{\BRbooknumberstyle{1}Esdras}%
  \setbooktitle{PrayerofManasseh}{Prayer of Manasseh}%
  \setbooktitle{PsalmCLI}{Psalm 151}%
  \setbooktitle{IIIMaccabees}{\BRbooknumberstyle{3}Maccabees}%
  \setbooktitle{IIEsdras}{\BRbooknumberstyle{2}Esdras}%
  \setbooktitle{IVMaccabees}{\BRbooknumberstyle{4}Maccabees}%
}
%    \end{macrocode}
% \end{macro}
%
% \begin{macro}{\brsblabbrvname}
%
%   Define new \cs{brsblabbrvname} macro which adjusts \cs{braltabbrvname}
%   abbreviations to bring them into line with SBL requirements.
%
%    \begin{macrocode}
\newcommand*\brsblabbrvname{%
  \braltabbrvname
  \brsbldefaults
%    \end{macrocode}
%
% \pkg{bibleref-parse} patches \cs{braltabbrvname} which overwrites
% definitions for |AddEsth|, |EpJer| and |AddDan|, so correct them in this
% case.
%
%    \begin{macrocode}
  \@ifpackageloaded{bibleref-parse}
    {\brsbl@setabbrv{AddEsth}{AdditionstoEsther}%
     \brsbl@setabbrv{EpJer}{EpistleofJeremiah}%
     \brsbl@setabbrv{AddDan}{AdditionstoDaniel}}
    {}%
%    \end{macrocode}
%
%    \begin{macrocode}
  \setbooktitle{IKingdoms}{\BRbooknumberstyle{1}Kgdms}%
  \setbooktitle{IIKingdoms}{\BRbooknumberstyle{2}Kgdms}%
  \setbooktitle{IIIKingdoms}{\BRbooknumberstyle{3}Kgdms}%
  \setbooktitle{IVKingdoms}{\BRbooknumberstyle{4}Kgdms}%
  \setbooktitle{Psalm}{Ps}%
  \setbooktitle{Psalms}{Pss}%
  \setbooktitle{Esther}{Esth}%
  \setbooktitle{Ecclesiastes}{\if@brsbl@qoh Qoh\else Eccl\fi}%
  \setbooktitle{SongofSongs}{\if@brsbl@cant Cant\else Song\fi}%
  \setbooktitle{Tobit}{Tob}%
  \setbooktitle{Judith}{Jdt}%
  \setbooktitle{AdditionstoEsther}{Add Esth}%
  \setbooktitle{Wisdom}{Wis}%
  \setbooktitle{Ecclesiasticus}{Sir}%
  \setbooktitle{Baruch}{Bar}%
  \setbooktitle{EpistleofJeremiah}{Ep Jer}%
  \setbooktitle{AdditionstoDaniel}{Add Dan}%
  \setbooktitle{PrayerofAzariah}{Pr Azar}%
  \setbooktitle{SongoftheThreeYoungMen}{Sg Three}%
  \setbooktitle{Susanna}{Sus}%
  \setbooktitle{BelandtheDragon}{Bel}%
  \setbooktitle{IMaccabees}{\BRbooknumberstyle{1}Macc}%
  \setbooktitle{IIMaccabees}{\BRbooknumberstyle{2}Macc}%
  \setbooktitle{IEsdras}{\BRbooknumberstyle{1}Esd}%
  \setbooktitle{PrayerofManasseh}{Pr Man}%
  \setbooktitle{PsalmCLI}{Ps 151}%
  \setbooktitle{IIIMaccabees}{\BRbooknumberstyle{3}Macc}%
  \setbooktitle{IIEsdras}{\BRbooknumberstyle{2}Esd}%
  \setbooktitle{IVMaccabees}{\BRbooknumberstyle{4}Macc}%
  \setbooktitle{Titus}{Titus}%
  \setbooktitle{Philemon}{Phlm}%
}
%    \end{macrocode}
% \end{macro}
%
% \begin{macro}{\brsblidxname}
%
%   Define new \cs{brsblindexname} macro which uses full book names in the
%   index.
%
%    \begin{macrocode}
\newcommand*\brsblidxname{%
  \brsbldefaults
  \setindexbooktitle{Gen}{Genesis}%
  \setindexbooktitle{Exod}{Exodus}%
  \setindexbooktitle{Lev}{Leviticus}%
  \setindexbooktitle{Num}{Numbers}%
  \setindexbooktitle{Deut}{Deuteronomy}%
  \setindexbooktitle{Josh}{Joshua}%
  \setindexbooktitle{Judg}{Judges}%
  \setindexbooktitle{Ruth}{Ruth}%
  \setindexbooktitle{1Sam}{\BRbooknumberstyle{1}Samuel}%
  \setindexbooktitle{1Kgdms}{\BRbooknumberstyle{1}Kingdoms}%
  \setindexbooktitle{2Sam}{\BRbooknumberstyle{2}Samuel}%
  \setindexbooktitle{2Kgdms}{\BRbooknumberstyle{2}Kingdoms}%
  \setindexbooktitle{1Kgs}{\BRbooknumberstyle{1}Kings}%
  \setindexbooktitle{3Kgdms}{\BRbooknumberstyle{3}Kingdoms}%
  \setindexbooktitle{2Kgs}{\BRbooknumberstyle{2}Kings}%
  \setindexbooktitle{4Kgdms}{\BRbooknumberstyle{4}Kingdoms}%
  \setindexbooktitle{1Chr}{\BRbooknumberstyle{1}Chronicles}%
  \setindexbooktitle{2Chr}{\BRbooknumberstyle{2}Chronicles}%
  \setindexbooktitle{Ezra}{Ezra}%
  \setindexbooktitle{Neh}{Nehemiah}%
  \setindexbooktitle{Ps}{Psalms}%
  \setindexbooktitle{Pss}{Psalms}%
  \setindexbooktitle{Prov}{Proverbs}%
  \setindexbooktitle{Eccl}{\if@brsbl@qoh Qoheleth\else Ecclesiastes\fi}%
  \setindexbooktitle{Qoh}{\if@brsbl@qoh Qoheleth\else Ecclesiastes\fi}%
  \setindexbooktitle{Song}{\if@brsbl@cant Canticles\else Song of Songs\fi}%
  \setindexbooktitle{Cant}{\if@brsbl@cant Canticles\else Song of Songs\fi}%
  \setindexbooktitle{Isa}{Isaiah}%
  \setindexbooktitle{Jer}{Jeremiah}%
  \setindexbooktitle{Lam}{Lamentations}%
  \setindexbooktitle{Ezek}{Ezekiel}%
  \setindexbooktitle{Dan}{Daniel}%
  \setindexbooktitle{Hos}{Hosea}%
  \setindexbooktitle{Joel}{Joel}%
  \setindexbooktitle{Amos}{Amos}%
  \setindexbooktitle{Obad}{Obadiah}%
  \setindexbooktitle{Jonah}{Jonah}%
  \setindexbooktitle{Mic}{Micah}%
  \setindexbooktitle{Nah}{Nahum}%
  \setindexbooktitle{Hab}{Habakkuk}%
  \setindexbooktitle{Zeph}{Zephaniah}%
  \setindexbooktitle{Hag}{Haggai}%
  \setindexbooktitle{Zech}{Zechariah}%
  \setindexbooktitle{Mal}{Malachi}%
%    \end{macrocode}
%
%    \begin{macrocode}
  \setindexbooktitle{Tob}{Tobit}%
  \setindexbooktitle{Jdt}{Judith}%
  \setindexbooktitle{AddEsth}{Additions to Esther}%
  \setindexbooktitle{Wis}{Wisdom}%
  \setindexbooktitle{Sir}{Sirach}%
  \setindexbooktitle{Bar}{Baruch}%
  \setindexbooktitle{EpJer}{Epistle of Jeremiah}%
  \setindexbooktitle{AddDan}{Additions to Daniel}%
  \setindexbooktitle{PrAzar}{Prayer of Azariah}%
  \setindexbooktitle{SgThree}{Song of the Three Young Men}%
  \setindexbooktitle{Sus}{Susanna}%
  \setindexbooktitle{Bel}{Bel and the Dragon}%
  \setindexbooktitle{1Macc}{\BRbooknumberstyle{1}Maccabees}%
  \setindexbooktitle{2Macc}{\BRbooknumberstyle{2}Maccabees}%
  \setindexbooktitle{1Esd}{\BRbooknumberstyle{1}Esdras}%
  \setindexbooktitle{PrMan}{Prayer of Manasseh}%
  \setindexbooktitle{Ps151}{Psalm 151}%
  \setindexbooktitle{3Macc}{\BRbooknumberstyle{3}Maccabees}%
  \setindexbooktitle{2Esd}{\BRbooknumberstyle{2}Esdras}%
  \setindexbooktitle{4Macc}{\BRbooknumberstyle{4}Maccabees}%
%    \end{macrocode}
%
%    \begin{macrocode}
  \setindexbooktitle{Matt}{Matthew}%
  \setindexbooktitle{Rom}{Romans}%
  \setindexbooktitle{1Cor}{\BRbooknumberstyle{1}Corinthians}%
  \setindexbooktitle{2Cor}{\BRbooknumberstyle{2}Corinthians}%
  \setindexbooktitle{Gal}{Galatians}%
  \setindexbooktitle{Eph}{Ephesians}%
  \setindexbooktitle{Phil}{Philippians}%
  \setindexbooktitle{Col}{Colossians}%
  \setindexbooktitle{1Thess}{\BRbooknumberstyle{1}Thessalonians}%
  \setindexbooktitle{2Thess}{\BRbooknumberstyle{2}Thessalonians}%
  \setindexbooktitle{1Tim}{\BRbooknumberstyle{1}Timothy}%
  \setindexbooktitle{2Tim}{\BRbooknumberstyle{2}Timothy}%
  \setindexbooktitle{Phlm}{Philemon}%
  \setindexbooktitle{Heb}{Hebrews}%
  \setindexbooktitle{Jas}{James}%
  \setindexbooktitle{1Pet}{\BRbooknumberstyle{1}Peter}%
  \setindexbooktitle{2Pet}{\BRbooknumberstyle{2}Peter}%
  \setindexbooktitle{1John}{\BRbooknumberstyle{1}John}%
  \setindexbooktitle{2John}{\BRbooknumberstyle{2}John}%
  \setindexbooktitle{3John}{\BRbooknumberstyle{3}John}%
  \setindexbooktitle{Rev}{Revelation}%
}
%    \end{macrocode}
% \end{macro}
%
% \begin{macro}{\brsbldefaults}
%
%   Set SBL default format and delimiters.
%
%    \begin{macrocode}
\newcommand*{\brsbldefaults}{%
  \renewcommand*{\BRbooknumberstyle}[1]{##1~}%
  \renewcommand*{\BRepistlenumberstyle}[1]{##1~}%
  \renewcommand*{\BRbookof}{}%
  \renewcommand*{\BRgospel}{}%
  \renewcommand*{\BRepistleto}{}%
  \renewcommand*{\BRepistletothe}{}%
  \renewcommand*{\BRepistleof}{}%
  \renewcommand*{\BRbooktitlestyle}[1]{##1}%
  \renewcommand*{\BRchapterstyle}[1]{##1}%
  \renewcommand*{\BRotherchapterstyle}[1]{(##1)}%
  \renewcommand*{\BRversestyle}[1]{##1}%
  \renewcommand*{\BRversesuffixstyle}[1]{##1}%
  \renewcommand*{\BRbkchsep}{ }%
  \renewcommand*{\BRchvsep}{:}%
  \renewcommand*{\BRchsep}{;}%
  \renewcommand*{\BRvrsep}{--}%
  \renewcommand*{\BRvsep}{,}%
  \renewcommand*{\BRperiod}{}%
}
%    \end{macrocode}
% \end{macro}
%
% \begin{macro}{\brsblot, \brsbldb, \brsblnt}
%
%   Name of categories for Old Testament, Deuterocanonical books and New
%   Testament books used in index.
%
%    \begin{macrocode}
\newcommand*{\brsblot}{%
  \if@brsbl@hebreworder
    Hebrew Bible/Old Testament%
  \else
    Old Testament/Hebrew Bible%
  \fi}
\newcommand*{\brsbldb}{Deuterocanonical Books}
\newcommand*{\brsblnt}{New Testament}
%    \end{macrocode}
% \end{macro}
%
% Set up index mapping if the |makeindexsorting| option is set to |true|.
%
%    \begin{macrocode}
\if@brsbl@makeindexsorting
%    \end{macrocode}
%
%    \begin{macrocode}
  \biblerefmap{Genesis}{10@\brsblot !01}
  \biblerefmap{Gen}{10@\brsblot !01}
  \biblerefmap{Exodus}{10@\brsblot !02}
  \biblerefmap{Exod}{10@\brsblot !02}
  \biblerefmap{Leviticus}{10@\brsblot !03}
  \biblerefmap{Lev}{10@\brsblot !03}
  \biblerefmap{Numbers}{10@\brsblot !04}
  \biblerefmap{Num}{10@\brsblot !04}
  \biblerefmap{Deuteronomy}{10@\brsblot !05}
  \biblerefmap{Deut}{10@\brsblot !05}
  \biblerefmap{Joshua}{10@\brsblot !06}
  \biblerefmap{Josh}{10@\brsblot !06}
  \biblerefmap{Judges}{10@\brsblot !07}
  \biblerefmap{Judg}{10@\brsblot !07}
  \biblerefmap{Ruth}{10@\brsblot !08}
  \biblerefmap{Samuel1}{10@\brsblot !09}
  \biblerefmap{Sam1}{10@\brsblot !09}
  \biblerefmap{Kingdoms1}{10@\brsblot !10}
  \biblerefmap{Kgdms1}{10@\brsblot !10}
  \biblerefmap{Samuel2}{10@\brsblot !11}
  \biblerefmap{Sam2}{10@\brsblot !11}
  \biblerefmap{Kingdoms2}{10@\brsblot !12}
  \biblerefmap{Kgdms2}{10@\brsblot !12}
  \biblerefmap{Kings1}{10@\brsblot !13}
  \biblerefmap{Kgs1}{10@\brsblot !13}
  \biblerefmap{Kingdoms3}{10@\brsblot !14}
  \biblerefmap{Kgdms3}{10@\brsblot !14}
  \biblerefmap{Kings2}{10@\brsblot !15}
  \biblerefmap{Kgs2}{10@\brsblot !15}
  \biblerefmap{Kingdoms4}{10@\brsblot !16}
  \biblerefmap{Kgdms4}{10@\brsblot !16}
  \biblerefmap{Chronicles1}{10@\brsblot !17}
  \biblerefmap{Chr1}{10@\brsblot !17}
  \biblerefmap{Chronicles2}{10@\brsblot !18}
  \biblerefmap{Chr2}{10@\brsblot !18}
  \biblerefmap{Ezra}{10@\brsblot !19}
  \biblerefmap{Nehemiah}{10@\brsblot !20}
  \biblerefmap{Neh}{10@\brsblot !20}
  \biblerefmap{Job}{10@\brsblot !21}
  \biblerefmap{Psalm}{10@\brsblot !22}
  \biblerefmap{Ps}{10@\brsblot !22}
  \biblerefmap{Psalms}{10@\brsblot !22}
  \biblerefmap{Pss}{10@\brsblot !22}
  \biblerefmap{Proverbs}{10@\brsblot !23}
  \biblerefmap{Prov}{10@\brsblot !23}
  \biblerefmap{Ecclesiastes}{10@\brsblot !24}
  \biblerefmap{Eccl}{10@\brsblot !24}
  \biblerefmap{Qoheleth}{10@\brsblot !24}
  \biblerefmap{Qoh}{10@\brsblot !24}
  \biblerefmap{Song of Songs}{10@\brsblot !25}
  \biblerefmap{Song}{10@\brsblot !25}
  \biblerefmap{Canticles}{10@\brsblot !25}
  \biblerefmap{Cant}{10@\brsblot !25}
  \biblerefmap{Isaiah}{10@\brsblot !26}
  \biblerefmap{Isa}{10@\brsblot !26}
  \biblerefmap{Jeremiah}{10@\brsblot !27}
  \biblerefmap{Jer}{10@\brsblot !27}
  \biblerefmap{Lamentations}{10@\brsblot !28}
  \biblerefmap{Lam}{10@\brsblot !28}
  \biblerefmap{Ezekiel}{10@\brsblot !29}
  \biblerefmap{Ezek}{10@\brsblot !29}
  \biblerefmap{Daniel}{10@\brsblot !30}
  \biblerefmap{Dan}{10@\brsblot !30}
  \biblerefmap{Hosea}{10@\brsblot !31}
  \biblerefmap{Hos}{10@\brsblot !31}
  \biblerefmap{Joel}{10@\brsblot !32}
  \biblerefmap{Amos}{10@\brsblot !33}
  \biblerefmap{Obadiah}{10@\brsblot !34}
  \biblerefmap{Obad}{10@\brsblot !34}
  \biblerefmap{Jonah}{10@\brsblot !35}
  \biblerefmap{Micah}{10@\brsblot !36}
  \biblerefmap{Mic}{10@\brsblot !36}
  \biblerefmap{Nahum}{10@\brsblot !37}
  \biblerefmap{Nah}{10@\brsblot !37}
  \biblerefmap{Habakkuk}{10@\brsblot !38}
  \biblerefmap{Hab}{10@\brsblot !38}
  \biblerefmap{Zephaniah}{10@\brsblot !39}
  \biblerefmap{Zeph}{10@\brsblot !39}
  \biblerefmap{Haggai}{10@\brsblot !40}
  \biblerefmap{Hag}{10@\brsblot !40}
  \biblerefmap{Zechariah}{10@\brsblot !41}
  \biblerefmap{Zech}{10@\brsblot !41}
  \biblerefmap{Malachi}{10@\brsblot !42}
  \biblerefmap{Mal}{10@\brsblot !42}
%    \end{macrocode}
%
%    \begin{macrocode}
  \biblerefmap{Tobit}{20@\brsbldb !01}
  \biblerefmap{Tob}{20@\brsbldb !01}
  \biblerefmap{Judith}{20@\brsbldb !02}
  \biblerefmap{Jdt}{20@\brsbldb !02}
  \biblerefmap{Additions to Esther}{20@\brsbldb !03}
  \biblerefmap{Add Esth}{20@\brsbldb !03}
  \biblerefmap{Wisdom}{20@\brsbldb !04}
  \biblerefmap{Wis}{20@\brsbldb !04}
  \biblerefmap{Sirach}{20@\brsbldb !05}
  \biblerefmap{Sir}{20@\brsbldb !05}
  \biblerefmap{Baruch}{20@\brsbldb !06}
  \biblerefmap{Bar}{20@\brsbldb !06}
  \biblerefmap{Epistle of Jeremiah}{20@\brsbldb !07}
  \biblerefmap{Ep Jer}{20@\brsbldb !07}
  \biblerefmap{Additions to Daniel}{20@\brsbldb !08}
  \biblerefmap{Add Dan}{20@\brsbldb !08}
  \biblerefmap{Prayer of Azariah}{20@\brsbldb !09}
  \biblerefmap{Pr Azar}{20@\brsbldb !09}
  \biblerefmap{Song of the Three Young Men}{20@\brsbldb !10}
  \biblerefmap{Sg Three}{20@\brsbldb !10}
  \biblerefmap{Susanna}{20@\brsbldb !11}
  \biblerefmap{Sus}{20@\brsbldb !11}
  \biblerefmap{Bel and the Dragon}{20@\brsbldb !12}
  \biblerefmap{Bel}{20@\brsbldb !12}
  \biblerefmap{Maccabees1}{20@\brsbldb !13}
  \biblerefmap{Macc1}{20@\brsbldb !13}
  \biblerefmap{Maccabees2}{20@\brsbldb !14}
  \biblerefmap{Macc2}{20@\brsbldb !14}
  \biblerefmap{Esdras1}{20@\brsbldb !15}
  \biblerefmap{Esd1}{20@\brsbldb !15}
  \biblerefmap{Prayer of Manasseh}{20@\brsbldb !16}
  \biblerefmap{Pr Man}{20@\brsbldb !16}
  \biblerefmap{Psalm 151}{20@\brsbldb !17}
  \biblerefmap{Ps 151}{20@\brsbldb !17}
  \biblerefmap{Maccabees3}{20@\brsbldb !18}
  \biblerefmap{Macc3}{20@\brsbldb !18}
  \biblerefmap{Esdras2}{20@\brsbldb !19}
  \biblerefmap{Esd2}{20@\brsbldb !19}
  \biblerefmap{Maccabees4}{20@\brsbldb !20}
  \biblerefmap{Macc4}{20@\brsbldb !20}
%    \end{macrocode}
%
%    \begin{macrocode}
  \biblerefmap{Matthew}{30@\brsblnt !01}
  \biblerefmap{Matt}{30@\brsblnt !01}
  \biblerefmap{Mark}{30@\brsblnt !02}
  \biblerefmap{Luke}{30@\brsblnt !03}
  \biblerefmap{John}{30@\brsblnt !04}
  \biblerefmap{Acts}{30@\brsblnt !05}
  \biblerefmap{Romans}{30@\brsblnt !06}
  \biblerefmap{Rom}{30@\brsblnt !06}
  \biblerefmap{Corinthians1}{30@\brsblnt !07}
  \biblerefmap{Cor1}{30@\brsblnt !07}
  \biblerefmap{Corinthians2}{30@\brsblnt !08}
  \biblerefmap{Cor2}{30@\brsblnt !08}
  \biblerefmap{Galatians}{30@\brsblnt !09}
  \biblerefmap{Gal}{30@\brsblnt !09}
  \biblerefmap{Ephesians}{30@\brsblnt !10}
  \biblerefmap{Eph}{30@\brsblnt !10}
  \biblerefmap{Philippians}{30@\brsblnt !11}
  \biblerefmap{Phil}{30@\brsblnt !11}
  \biblerefmap{Colossians}{30@\brsblnt !12}
  \biblerefmap{Col}{30@\brsblnt !12}
  \biblerefmap{Thessalonians1}{30@\brsblnt !13}
  \biblerefmap{Thess1}{30@\brsblnt !13}
  \biblerefmap{Thessalonians2}{30@\brsblnt !14}
  \biblerefmap{Thess2}{30@\brsblnt !14}
  \biblerefmap{Timothy1}{30@\brsblnt !15}
  \biblerefmap{Tim1}{30@\brsblnt !15}
  \biblerefmap{Timothy2}{30@\brsblnt !16}
  \biblerefmap{Tim2}{30@\brsblnt !16}
  \biblerefmap{Titus}{30@\brsblnt !17}
  \biblerefmap{Philemon}{30@\brsblnt !18}
  \biblerefmap{Phlm}{30@\brsblnt !18}
  \biblerefmap{Hebrews}{30@\brsblnt !19}
  \biblerefmap{Heb}{30@\brsblnt !19}
  \biblerefmap{James}{30@\brsblnt !20}
  \biblerefmap{Jas}{30@\brsblnt !20}
  \biblerefmap{Peter1}{30@\brsblnt !21}
  \biblerefmap{Pet1}{30@\brsblnt !21}
  \biblerefmap{Peter2}{30@\brsblnt !22}
  \biblerefmap{Pet2}{30@\brsblnt !22}
  \biblerefmap{John1}{30@\brsblnt !23}
  \biblerefmap{John2}{30@\brsblnt !24}
  \biblerefmap{John3}{30@\brsblnt !25}
  \biblerefmap{Jude}{30@\brsblnt !26}
  \biblerefmap{Revelation}{30@\brsblnt !27}
  \biblerefmap{Rev}{30@\brsblnt !27}
%    \end{macrocode}
%
%    \begin{macrocode}
\fi
%    \end{macrocode}
%
% Define new \texttt{sbl} style for \pkg{bibleref}.
%
%    \begin{macrocode}
\newbiblerefstyle{sbl}{%
  \biblerefstyle{default}%
  \brsblabbrvname
  \brsblidxname
}
%    \end{macrocode}
%
% Load new \texttt{sbl} style.
%
%    \begin{macrocode}
\biblerefstyle{sbl}
%    \end{macrocode}
%
%    \begin{macrocode}
\endinput
%    \end{macrocode}
%
%    \begin{macrocode}
%</package>
%    \end{macrocode}
