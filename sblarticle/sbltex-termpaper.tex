\documentclass{sblarticle}

\addbibresource{biblatex-sbl.bib}

\begin{document}

\title{Title of Paper}
\author{[Your Name]}
\maketitle

\printbiblist[heading=biblistintoc]{abbreviations}

The top margin is two inches for the first page only. There are two blank
lines between the title and the text (or subheading if there is one). The
left, right, top, and bottom margins are one inch. The first pages of chapters
are formatted like the primary heading.

Indent the first line of subsequent paragraphs. All main text should be set in
a standard 12-point font such as Times-New-Roman.

The \texttt{sblarticle} class uses the TeX Gyre Termes font by default, but an
alternative can be specified.

\section[First-level subheading (indent three spaces, no dots); titles that
exceed one line must also be indented for subsequent lines]{First-Level
Subheading}

Keep two blank lines between the text of the preceding section and a
subheading, regardless of the level. A first level subheading is centered,
bold, and capitalized headline style.

\subsection[Second-level subheadings (indent three spaces, no
dots)]{Second-Level Subheading}

There are two blank lines between the text of the preceding section and the
subheading. A second-level subheading is centered and capitalized headline
style.

\subsubsection{Third-Level Subheading}

A third level subheading is on the left margin, in bold, italics, and
capitalized headline style. A heading should never be the last text on a page.
If necessary, add extra blank space at the end of the page and begin the
following page with a heading.

\paragraph{Fourth-Level Subheading}

A fourth-level subheading is on the left margin, capitalized headline style.

The page numbers for the noninitial pages of the paper (or chapter) are
located at the top right corner. The text of the body of the paper is
double-spaced except for blocked
quotations.
\begin{quote}
  This is a blocked quotation. It should consist of five or more lines of text
  and be indented one-half inch. Block quotations should be single-spaced. No
  quotation marks are used at the beginning or the end of the quote. Double
  quotation marks within the original matter are retained. The blocked quote
  is set off by a regular double space before and after the quote. Note that
  regular spacing resumes after the end of the quotation.\footnote{The first
  line of a footnote is indented one-half inch. A 10-point font is acceptable.
  Footnotes, unlike the main text of the paper, should be single-spaced.}
\end{quote}
After a block quotation, return to double-spaced text justified to the left
margin until you finish the paragraph.

Footnotes at the bottom of the page are separated by a two-inch
rule.\footnote{There should be a blank line between each note and a blank en
space between the number and the first word of the note.} Maintain subsequent
numbering in notes. Make sure a footnote and the text to which it refers are
on the same page.

\section{Citations}

Citations should be referenced using \verb+\autocite+ or \verb+\autocites+ for
single volume resources and \verb+\avolcite+ or \verb+\avolcites+ for
multi-volume resources. This will place the citation in a footnote.
\verb+\parencite+, \verb+\parencites+, \verb+\pvolcite+, and \verb+\pvolcites+
place citations in parentheses.

For example, citing a classical primary source \ptranscite{tacitus:ann};
citing a lexicon \autocite[\foreignlanguage{polytonicgreek}{παρρησία}]{BDAG};
citing volume 2 of a multi-volume commentary
\avolcite{2}[125]{dahood:1965-1970}; citing two resources by the same author
in a single footnote \autocites[504]{harrington:1970}[241]{harrington:1986};
citing a resource for the second time \autocite[505]{harrington:1970}; citing
an article in an edited collection \autocite{collins:1986}; citing another
article in the same edited collection \autocite{attridge:1986}.

\section{Greek and Hebrew}

If available, \texttt{sblarticle} will use the SBL fonts for Greek and Hebrew
with Unicode engines otherwise the class falls back to Alegreya for Greek and
Frank Ruehl CLM for Hebrew. The class sets up the \texttt{babel} package for
\texttt{polytonicgreek} and \texttt{hebrew}. Other languages can be loaded
manually or on the fly with \texttt{lualatex}.

I strongly recommend using the \texttt{luatex} engine for multilingual work,
especially for right-to-left languages. Inline Greek and Hebrew need to be
marked up with \texttt{xetex} and \texttt{pdftex}. With \texttt{pdftex} Hebrew
pointing support is limited and cantillation marks are not supported at all.
\begin{quotation}
  \selectlanguage{polytonicgreek}
  Ἐν ἀρχῇ ἦν ὁ λόγος, καὶ ὁ λόγος ἦν πρὸς τὸν θεόν, καὶ θεὸς ἦν ὁ λόγος.
  οὗτος ἦν ἐν ἀρχῇ πρὸς τὸν θεόν.
\end{quotation}
\begin{quotation}
  \selectlanguage{hebrew}
  בּראשׁית בּרא אלהים את השּׁמים ואת הארץ׃ והארץ היתה תהוּ ובהוּ וחשׁך על־פּני תהום
  ורוּח אלהים מרחפת על־פּני המּים׃
\end{quotation}
Inline Greek (\foreignlanguage{polytonicgreek}{Ἐν ἀρχῇ}) and Hebrew
(\foreignlanguage{hebrew}{בּראשׁית}) are also possible.

\nocite{NIDNTT, Jastrow, DMBI, mclay:2006, oday:intertextuality, rad:1990}

\printbibliography[heading=bibintoc]
\end{document}

